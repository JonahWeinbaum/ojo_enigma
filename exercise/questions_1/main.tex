\documentclass{article}
\usepackage{amsmath,amssymb,amsthm, mathtools}
\usepackage{stmaryrd}
\usepackage{enumitem}
 \usepackage{layout}
 \usepackage[scale=0.75,top=1cm]{geometry}

\newtheorem{problem}{}

\begin{document}

\title{Question 1}
\author{Jonah Weinbaum}
\date{\today}
\maketitle

\section{Functions}

Here are some useful definitions regarding functions. Consider a function from a set $A$ to a set $B$ which we will denote $f: A\to B$.
\\\\
\begin{itemize}
\item We call $A$ the \textbf{domain} of $f$.
\item We call $B$ the \textbf{codomain} of $f$.
  \item For a subset $X\subset{A}$ we call the set $f(X) \coloneq \{b\in B \text{ }\vert\text{ }\exists\text{ }a\in{A}\text{ s.t }b=f(a)\}$ the \textbf{image} of $X$ under $f$.
\item We call the set $f(A)$ the \textbf{range} or \textbf{image} of $f$.
\item For a subset $Y\subseteq B$ we call the set $f^{-1}(Y) \coloneq \{a\in A\text{ }\vert\text{ }f(a) \in Y\}$ the \textbf{preimage} or \textbf{inverse image} of $Y$ under $f$.
\item We say $f$ is \textbf{injective} (is an \textbf{injection}) if $\forall\text{ }x_1, x_2\in {A}$ we have $x_1\ne x_2 \Rightarrow f(x_1) \ne f(x_2)$.
\item We say $f$ is \textbf{surjective} (is a \textbf{surjection}) if $\forall\text{ }y\in B$ we have $\exists\text{ }x\in A$ s.t. $f(x) = y$.
  \item We say $f$ is \textbf{bijective} (is a \textbf{bijection}) if it is both injective and surjective.
\end{itemize}

\begin{itemize}
\item [1.] For each function, show if the function is injective, surjective, or both.
  \begin{itemize}
    \item [a.]
    \begin{align*}
    f: \mathbb{R} &\to \mathbb{R}\\
    x &\mapsto x^2
    \end{align*}
    \item [b.]
    \begin{align*}
    f: \mathbb{R} &\to \mathbb{R}\\
    x &\mapsto (x+1)^3
    \end{align*}
    \item [c.]
    \begin{align*}
    f: \mathbb{R} &\to \mathbb{R}\\
      x &\mapsto |x|
    \end{align*}
    \item [d.]
    \begin{align*}
    f: \mathbb{R} &\to \mathbb{R}\\
    x &\mapsto 2x+1
    \end{align*}
  \end{itemize}
\item [2.] Let $f: A\to B$ and $g: B\to C$ be injections. Show that
  \begin{align*}
    g\circ f: A \to C
  \end{align*}
  is an injection.
\item [3.] Let $f: A\to B$ and $g: B\to C$ be surjections. Show that
  \begin{align*}
    g\circ f: A \to C
  \end{align*}
  is a surjection.
\item [4.] Let $f: A\to B$ and $g: B\to C$ be bijections. Show that
  \begin{align*}
    g\circ f: A \to C
  \end{align*}
  is a bijection.
\item [5.] Give an example of an injection $f:A \to B$ and a surjection $g: B \to C$ such that
  \begin{align*}
    g\circ f: A \to B
  \end{align*}
  is not surjective or injective?
\end{itemize}

Let $f: A \to B$ be a function. For a subset $X\subseteq{A}$ we defined the \textbf{restriction} of $f$ to $X$ as
\begin{align*}
  f|_X: X &\to B\\
  x &\mapsto f(x)
\end{align*}
\begin{itemize}
\item [6.] If $f: A \to B$ is an injection and $X\subseteq{A}$ show that $f|_X$ is an injection.
\item [7.] If $f: A \to B$ is an injection. Show that
  \begin{align*}
    g: A &\to f(A)\\
    x &\mapsto f(x)
  \end{align*}
  is a bijection.
\end{itemize}
\section{The Integers Modulo $n$ $(\mathbb{Z}/n\mathbb{Z})$}

Let $n\in\mathbb{Z}$. Define a relation on $\mathbb{Z}$ by
\begin{center}
  $a\sim_{n} b \iff \exists\text{ }k\in\mathbb{Z}\text{ s.t. }b-a = kn$
\end{center}

\begin{itemize}
\item [1.] Prove that $\sim_{n}$ is an equivalence relation.
\item [2.] How many distinct equivalence classes are there in $\mathbb{Z}/\sim_{n}$?
\end{itemize}

We call the set $\mathbb{Z}/\sim_{n}$  the \textbf{integers modulo $n$} and it is denoted $\mathbb{Z}/n\mathbb{Z}$.
We can turn this into a group by adding a binary operation.

\subsection{The Additive Group $(\mathbb{Z}/n\mathbb{Z}, +)$}

We can define the additive binary operation on $\mathbb{Z}/n\mathbb{Z}$ as follows, $\forall\text{ }a,b\in\mathbb{Z}/n\mathbb{Z}$
\begin{align*}
  \overline{a} + \overline{b} &\coloneq \overline{a+b}\\
\end{align*}
  Let us look at a specific example.
  \\\\Consider $\mathbb{Z}/5\mathbb{Z} = \{\overline{0}, \overline{1}, \overline{2}, \overline{3}, \overline{4}\}$ equipped with the above binary operation
  \begin{itemize}
  \item [3.] Show that $(\mathbb{Z}/5\mathbb{Z}, +)$ is a group.
  \item [4.] What is $\overline{3} + \overline{4}$?
  \item [5.] What is $\overline{3} + \overline{2}$?
  \item [6.] What is $(\overline{3})^5$?
  \item [7.] What is $(\overline{2})^{-1}$
  \end{itemize}

\subsection{The Multiplicative Group $(\mathbb{Z}/n\mathbb{Z}, \cdot)$}

We can define the multiplicative binary operation on $\mathbb{Z}/n\mathbb{Z}$ as follows, $\forall\text{ }a,b\in\mathbb{Z}/n\mathbb{Z}$
\begin{align*}
  \overline{a} \cdot \overline{b} &\coloneq \overline{a\cdot b}\\
\end{align*}
  Let us look at a specific example.
  \\\\Consider $(\mathbb{Z}/5\mathbb{Z})^\times = \{\overline{1}, \overline{2}, \overline{3}, \overline{4}\}$ (notice we don't include $\overline{0}$ since it has no inverse) equipped with the above binary operation
  \begin{itemize}
  \item [8.] What is the identity?
  \item [9.] What is $\overline{3} \cdot \overline{4}$?
  \item [10.] What is $\overline{3} \cdot \overline{2}$?
  \item [11.] What is $(\overline{3})^5$?
  \item [12.] What is $(\overline{2})^{-1}$
  \end{itemize}

  \section{Group Theory}

  A \textbf{binary operation} $\star$ on a set $G$ is a function
  \begin{align*}
    \star: G \times G \to G
  \end{align*}
  and, for convenience, for any $a,b\in{G}$ we write $\star(a,b)$ as $a\star{b}$.
  \\\\A \textbf{group} is a set $G$ equipped with a binary operation $\star$ on $G$ such that
  \begin{itemize}
  \item $\forall\text{ }a,b\in{G}$ we have $a\star(b\star{c}) = (a\star{b})\star{c}$ (that is $\star$ is \textbf{associative})
  \item There exists an element $e\in{G}$ (which we call the \textbf{identity}) such that $\forall\text{ }a\in{G}$ we have
    \begin{center}
      $e\star{a}=a\star{e}=a$
    \end{center}
  \item For each element $a\in{G}$ there is an element $a^{-1}\in{G}$ (which we call the \textbf{inverse} of $a$) such that
    \begin{center}
      $a\star{a^{-1}}=a^{-1}\star{a}=e$
    \end{center}
  \end{itemize}
  We denote the group $(G, \star)$.
  \begin{itemize}
  \item [1.] Let $(G,\star)$ be a group. Show that there exists \emph{only one} element $e\in{G}$ such that
    \begin{center}
      $\forall\text{ }a\in{G}\text{ we have }a\star{e}=e\star{a}=a$
    \end{center}
    That is, we know the identity exists by definition, but show it is unique.
  \item[2.] Let $(G, \star)$ be a group. Let $a\in{G}$. Show that there exists \emph{only one} element $a^{-1}\in{G}$ such that
    \begin{center}
      $a\star a^{-1} = a^{-1}\star {a} = e$
    \end{center}
    That is, we know the inverse exists by definition, but show it is unique.
  \item [3.] Let $(G, \star)$ be a group. Let $a\in{G}$. Show that
    \begin{center}
      $(a^{-1})^{-1} = a$
    \end{center}
  \item [4.] Let $(G, \star)$ be a group. Let $a,b\in{G}$. Show that
    \begin{center}
      $(a\star{b})^{-1} = (b^{-1})\star(a^{-1})$
    \end{center}
  \item [5.] Which of the following sets are groups with respect to the binary operation of addition:
    \begin{itemize}
    \item The set of rational numbers in lowest terms (reduced) whose denominators are odd.
    \item The set of rational numbers in lowest terms (reduced) whose denominators are even.
    \item The set $\{q\in\mathbb{Q}\text{ }\vert\text{ }|q| < 1\}$.
    \item The set $\{q\in\mathbb{Q}\text{ }\vert\text{ }|q| \ge 1\} \cup \{0\}$.
    \item The set $\{\frac{a}{b}\in\mathbb{Q}\text{ }\vert\text{ }b=1\text{ or }b=2\}$
    \end{itemize}
  \end{itemize}
  
\end{document}
