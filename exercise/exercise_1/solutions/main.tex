\documentclass{article}
\usepackage{amsmath,amssymb,amsthm, mathtools}
\usepackage{stmaryrd}
\usepackage{enumitem}
\usepackage{layout}
\usepackage{comment}
\usepackage[scale=0.75,top=1cm]{geometry}

\newenvironment{theorem}[2][Theorem]{\begin{trivlist}
\item[\hskip \labelsep {\bfseries #1}\hskip \labelsep {\bfseries #2.}]}{\end{trivlist}}
\newenvironment{lemma}[2][Lemma]{\begin{trivlist}
\item[\hskip \labelsep {\bfseries #1}\hskip \labelsep {\bfseries #2.}]}{\end{trivlist}}
\newenvironment{exercise}[2][Exercise]{\begin{trivlist}
\item[\hskip \labelsep {\bfseries #1}\hskip \labelsep {\bfseries #2.}]}{\end{trivlist}}
\newenvironment{reflection}[2][Reflection]{\begin{trivlist}
\item[\hskip \labelsep {\bfseries #1}\hskip \labelsep {\bfseries #2.}]}{\end{trivlist}}
\newenvironment{proposition}[2][Proposition]{\begin{trivlist}
\item[\hskip \labelsep {\bfseries #1}\hskip \labelsep {\bfseries #2.}]}{\end{trivlist}}
\newenvironment{corollary}[2][Corollary]{\begin{trivlist}
\item[\hskip \labelsep {\bfseries #1}\hskip \labelsep {\bfseries #2.}]}{\end{trivlist}}
\newenvironment{problem}[2][Problem]{\begin{trivlist}
\item[\hskip \labelsep {\bfseries #1}\hskip \labelsep {\bfseries #2.}]}{\end{trivlist}}
\newenvironment{definition}[2][Definition]{\begin{trivlist}
\item[\hskip \labelsep {\emph{#1}}\hskip \labelsep {#2.}]}{\end{trivlist}}

\begin{document}

\title{Exercise 1}
\author{Jonah Weinbaum}
\date{\today}
\maketitle

\begin{itemize}
    \item [1.] Show that an element of a finite group is of finite order. 
        \\That is, if $(G, \cdot)$ is such that $|G| < \infty$ then $\forall\text{ }x\in{G}$ then $\text{ord}(x) = k$ for some $k\in\mathbb{N}$. The $\operatorname{ord}(x)$ is the smallest positive integer $k$ such that $x^k = e$. If no such $k$ exists we say $\operatorname{ord}(x) = \infty$.
    \item [2.]Let $S_n$ be the symmetric group on $n$ elements. Let $\sigma\in S_n$. Show that $\sim_{\sigma}$ defined by 
    \begin{center}
        $i\sim_\sigma j \iff \exists\text{ }k\in\mathbb{Z}\colon\sigma^k(i)=j$
    \end{center}
    is an equivalence relation on $\mathbb{N}_n$.
    \begin{proof}
      \begin{enumerate}
      \item Fix $i\in\mathbb{N}_n$. Then $\sigma^0(i) = i$ so $i \sim_\sigma i$.
      \item Fix $i,j\in\mathbb{N}_n$ s.t. $i\sim_\sigma j$. Then $\exists\text{ }k\in\mathbb{Z}$ s.t. $\sigma^k(i) = j$. Then we have
        \begin{align*}
          \sigma^{-k}(j) &= \sigma^{-k}(\sigma^k(i)) \\
          &= i
        \end{align*}
        Then $j\sim_\sigma{i}$
      \item Fix $x,y,z \in\mathbb{N}_n$ s.t. $x\sim_\sigma y$ and $y\sim_\sigma z$. Then
        \begin{center}
          $\exists \text{ } l\in\mathbb{Z}$ s.t. $\sigma^l(x) = y$\\
          $\exists \text{ } r\in\mathbb{Z}$ s.t. $\sigma^r(y) = z$
        \end{center}
        Then we have
        \begin{align*}
          \sigma^{l+r}(x) &= \sigma^r(\sigma^l(x))\\
                          &= \sigma^r(y)\\
                          &= z
        \end{align*}
        Then $x\sim_\sigma z$.
      \end{enumerate}
      Then $\sim_\sigma$ is an equivalence relation on $\mathbb{N}_n$. 
    \end{proof}
    \begin{definition}
      \\If $\sim$ is an equivalence relation on a set $S$. Then the \textbf{equivalence class} of $x\in S$ is the set
      \begin{center}
        $\overline{x} = [x] \coloneq \{y\in S\text{ }\vert\text{ } x\sim y\}$
      \end{center}
    \end{definition}
    \begin{definition}
      \\If $\sim$ is an equivalence relation on a set $S$. Then the \textbf{quotient of $S$ mod $\sim$} is the set
      \begin{center}
        $S/\sim\text{ }\coloneq \{\overline{x}\text{ }\vert\text{ }x\in S\}$
      \end{center}
    \end{definition}
    \item [3.]Let $\mathbb{N}_n/\sim_\sigma = \{E_1,\dots,E_m\}$ be the quotient set of $\mathbb{N}_n$ under $\sim_\sigma$. Show that 
    \begin{center}
    \begin{align*}
        \phi_i: E_i &\rightarrow E_i
        \\x &\mapsto \sigma(x)
    \end{align*}
    \end{center}
    is a well-defined permutation on $E_i$.
    \begin{proof}
      Fix $x\in E_i$. Then $\sigma(x) \sim_\sigma x$ by definition of $\sim_\sigma$ so $\sigma(x)\in E_i$ since $E_i$ is the equivalence class of $x$. Then $\phi_i$ is well-defined.
      Furthermore, $\phi_i$ is a permutation since it maintains injectivity because it is the restriction of an injection. Then this is an injection on a finite domain and range of equal size and is thus a bijection
      and thus a permutation on $E_i$. 
    \end{proof}
    \item [4.] Let 
    \begin{center}
    \begin{align*}
        \sigma_i: \mathbb{N}_n &\rightarrow \mathbb{N}_n
        \\x &\mapsto \begin{cases} 
      \sigma(x) & x\in E_i \\
      x & x\notin E_i
    \end{cases}
    \end{align*}
    \end{center}
    It follows from $(3)$ that $\sigma_i$ is a permutation on $\mathbb{N}_n$. Show that $E_i$ and $E_j$ are disjoint for $i\ne j$.
    \item [5.] Let $i\in\mathbb{N}_m$, show that $\sigma_i$ is a cycle. This can be reduced to showing that
    \begin{center}
        $E_i = \{x, \sigma_i(x), \dots \sigma_i^{k-1}(x)\}$ for some $k\in\mathbb{N}$ and $x\in E_i$.
    \end{center}
    \item [6.] Show that $\sigma = \sigma_1\sigma_2\dots\sigma_m$
    
    \item [7.] \[
        \sigma = \begin{pmatrix}
        1 & 2 & 3 & 4 & 5 & 6 \\
        1 & 3 & 2 & 5 & 6 & 4
        \end{pmatrix}
        \]
    Give $\sigma$ in its disjoint cycle notation.
    \item [8.] For all $\sigma\in S_n$ let $\sigma$ be expressed in cycle notation as a composite of $n$ cycles containing $m_1,\dots,m_n$ elements where each $m_i \ge 2$. 
    \text{}\\\\Let the $P: S_n\rightarrow \{1, -1\}$ be defined as follows:
    \begin{center}
        $\forall\text{ }\sigma\in S_n\colon P(\sigma) = \prod_{i=1}^n(-1)^{m_i-1}$
    \end{center}
    where $P(I_{S_n}) \coloneq 1$. 
    \text{}\\\\Show that for all transpositions $\tau\in S_n$ we have $P(\sigma\circ\tau) = -P(\sigma)$.
    \item [9.] Let $\sigma\in S_n$ be expressible as the composition of $r$ transpositions\text{}\\\\Show by induction that
    \begin{center}
        $P(\sigma) = (-1)^r$
    \end{center}
    \item [10.] Let $\sigma\in S_n$. Show that if $\sigma$ is expressible as a product of an even number of transpositions, then it cannot also be expressible as the product of an odd number of transpositions.
    \item [11.] By $(3)-(5)$ the following is now a well defined function:
    \begin{center}
        For a permutation $\sigma\in S_n$, if we decompose $\sigma$ into $k$ transpositions, then the \textbf{sign} or \textbf{parity} of $\sigma$ is \text{}\\
        $\operatorname{sgn}(\sigma) \coloneq\begin{cases} 
      1 & k\text{ is even} \\
      -1 & k\text{ is odd}
    \end{cases}$
    \end{center}
    Note that $P$ and $\operatorname{sgn}$ are equivalent
    \text{}\\\\Show the following regarding $\sigma, \rho\in S_n$
    \begin{enumerate}
        \item $\operatorname{sgn}(\sigma) = 1$ and $\operatorname{sgn}(\rho) = 1$ implies $\operatorname{sgn}(\rho\circ\sigma) = 1$
        \item $\operatorname{sgn}(\sigma) = -1$ and $\operatorname{sgn}(\rho) = -1$ implies $\operatorname{sgn}(\rho\circ\sigma) = 1$
        \item $\operatorname{sgn}(\sigma) = 1$ and $\operatorname{sgn}(\rho) = -1$ implies $\operatorname{sgn}(\rho\circ\sigma) = -1$
    \end{enumerate}
\end{itemize}


\end{document}
