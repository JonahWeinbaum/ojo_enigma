\documentclass{article}
\usepackage{amsmath,amssymb,amsthm, mathtools}
\usepackage{stmaryrd}
\usepackage{enumitem}
 \usepackage{layout}
 \usepackage[scale=0.75,top=1cm]{geometry}

\newtheorem{problem}{}

\begin{document}

\title{Exercise 1}
\author{Jonah Weinbaum}
\date{\today}
\maketitle

\begin{enumerate}
    \item Show that an element of a finite group is of finite order. 
        \\That is, if $(G, \cdot)$ is such that $|G| < \infty$ then $\forall\text{ }x\in{G}$ then $\text{ord}(x) = k$ for some $k\in\mathbb{N}$. The $\operatorname{ord}(x)$ is the smallest positive integer $k$ such that $x^k = e$. If no such $k$ exists we say $\operatorname{ord}(x) = \infty$.
    \item Let $S_n$ be the symmetric group on $n$ elements. Let $\sigma\in S_n$. Show that $\sim_{\sigma}$ defined by 
    \begin{center}
        $i\sim_\sigma j \iff \exists\text{ }k\in\mathbb{Z}\colon\sigma^k(i)=j$
    \end{center}
    is an equivalence relation.
    \item Let $\mathbb{N}_n/\sim_\sigma = \{E_1,\dots,E_m\}$ be the quotient set of $\mathbb{N}_n$ under $\sim_\sigma$. Show that 
    \begin{center}
    \begin{align*}
        \phi_i: E_i &\rightarrow E_i
        \\x &\mapsto \sigma(x)
    \end{align*}
    \end{center}
    is a well-defined permutation on $E_i$.
    \item Let 
    \begin{center}
    \begin{align*}
        \sigma_i: \mathbb{N}_n &\rightarrow \mathbb{N}_n
        \\x &\mapsto \begin{cases} 
      \sigma(x) & x\in E_i \\
      x & x\notin E_i
    \end{cases}
    \end{align*}
    \end{center}
    It follows from $(3)$ that $\sigma_i$ is a permutation on $\mathbb{N}_n$. Show that $E_i$ and $E_j$ are disjoint for $i\ne j$.
    \item Let $i\in\mathbb{N}_m$, show that $\sigma_i$ is a cycle. This can be reduced to showing that
    \begin{center}
        $E_i = \{x, \sigma_i(x), \dots \sigma_i^{k-1}(x)\}$ for some $k\in\mathbb{N}$ and $x\in E_i$.
    \end{center}
    \item Show that $\sigma = \sigma_1\sigma_2\dots\sigma_m$
    \item From $(2)-(6)$ we have shown that $\sigma$ can be expressed as a product of disjoint cycles. As it is out of scope we will \emph{not} show (but you should know) that this expression as a product of disjoint cycles is unique up to the ordering of the cycles.
    \item \[
        \sigma = \begin{pmatrix}
        1 & 2 & 3 & 4 & 5 & 6 \\
        1 & 3 & 2 & 5 & 6 & 4
        \end{pmatrix}
        \]
    Give $\sigma$ in its disjoint cycle notation.
    \item For all $\sigma\in S_n$ let $\sigma$ be expressed in cycle notation as a composite of $n$ cycles containing $m_1,\dots,m_n$ elements where each $m_i \ge 2$. 
    \text{}\\\\Let the $P: S_n\rightarrow \{1, -1\}$ be defined as follows:
    \begin{center}
        $\forall\text{ }\sigma\in S_n\colon P(\sigma) = \prod_{i=1}^n(-1)^{m_i-1}$
    \end{center}
    where $P(I_{S_n}) \coloneq 1$. 
    \text{}\\\\Show that for all transpositions $\tau\in S_n$ we have $P(\sigma\circ\tau) = -P(\sigma)$.
    \item Let $\sigma\in S_n$ be expressible as the composition of $r$ transpositions\text{}\\\\Show by induction that
    \begin{center}
        $P(\sigma) = (-1)^r$
    \end{center}
    \item Let $\sigma\in S_n$. Show that if $\sigma$ is expressible as a product of an even number of transpositions, then it cannot also be expressible as the product of an odd number of transpositions.
    \item By $(3)-(5)$ the following is now a well defined function:
    \begin{center}
        For a permutation $\sigma\in S_n$, if we decompose $\sigma$ into $k$ transpositions, then the \textbf{sign} or \textbf{parity} of $\sigma$ is \text{}\\
        $\operatorname{sgn}(\sigma) \coloneq\begin{cases} 
      1 & k\text{ is even} \\
      -1 & k\text{ is odd}
    \end{cases}$
    \end{center}
    Note that $P$ and $\operatorname{sgn}$ are equivalent
    \text{}\\\\Show the following regarding $\sigma, \rho\in S_n$
    \begin{enumerate}
        \item $\operatorname{sgn}(\sigma) = 1$ and $\operatorname{sgn}(\rho) = 1$ implies $\operatorname{sgn}(\rho\circ\sigma) = 1$
        \item $\operatorname{sgn}(\sigma) = -1$ and $\operatorname{sgn}(\rho) = -1$ implies $\operatorname{sgn}(\rho\circ\sigma) = 1$
        \item $\operatorname{sgn}(\sigma) = 1$ and $\operatorname{sgn}(\rho) = -1$ implies $\operatorname{sgn}(\rho\circ\sigma) = -1$
    \end{enumerate}
\end{enumerate}


\end{document}
